% Options for packages loaded elsewhere
\PassOptionsToPackage{unicode}{hyperref}
\PassOptionsToPackage{hyphens}{url}
%
\documentclass[
  man]{apa6}
\usepackage{amsmath,amssymb}
\usepackage{iftex}
\ifPDFTeX
  \usepackage[T1]{fontenc}
  \usepackage[utf8]{inputenc}
  \usepackage{textcomp} % provide euro and other symbols
\else % if luatex or xetex
  \usepackage{unicode-math} % this also loads fontspec
  \defaultfontfeatures{Scale=MatchLowercase}
  \defaultfontfeatures[\rmfamily]{Ligatures=TeX,Scale=1}
\fi
\usepackage{lmodern}
\ifPDFTeX\else
  % xetex/luatex font selection
\fi
% Use upquote if available, for straight quotes in verbatim environments
\IfFileExists{upquote.sty}{\usepackage{upquote}}{}
\IfFileExists{microtype.sty}{% use microtype if available
  \usepackage[]{microtype}
  \UseMicrotypeSet[protrusion]{basicmath} % disable protrusion for tt fonts
}{}
\makeatletter
\@ifundefined{KOMAClassName}{% if non-KOMA class
  \IfFileExists{parskip.sty}{%
    \usepackage{parskip}
  }{% else
    \setlength{\parindent}{0pt}
    \setlength{\parskip}{6pt plus 2pt minus 1pt}}
}{% if KOMA class
  \KOMAoptions{parskip=half}}
\makeatother
\usepackage{xcolor}
\usepackage{graphicx}
\makeatletter
\def\maxwidth{\ifdim\Gin@nat@width>\linewidth\linewidth\else\Gin@nat@width\fi}
\def\maxheight{\ifdim\Gin@nat@height>\textheight\textheight\else\Gin@nat@height\fi}
\makeatother
% Scale images if necessary, so that they will not overflow the page
% margins by default, and it is still possible to overwrite the defaults
% using explicit options in \includegraphics[width, height, ...]{}
\setkeys{Gin}{width=\maxwidth,height=\maxheight,keepaspectratio}
% Set default figure placement to htbp
\makeatletter
\def\fps@figure{htbp}
\makeatother
\setlength{\emergencystretch}{3em} % prevent overfull lines
\providecommand{\tightlist}{%
  \setlength{\itemsep}{0pt}\setlength{\parskip}{0pt}}
\setcounter{secnumdepth}{-\maxdimen} % remove section numbering
% Make \paragraph and \subparagraph free-standing
\ifx\paragraph\undefined\else
  \let\oldparagraph\paragraph
  \renewcommand{\paragraph}[1]{\oldparagraph{#1}\mbox{}}
\fi
\ifx\subparagraph\undefined\else
  \let\oldsubparagraph\subparagraph
  \renewcommand{\subparagraph}[1]{\oldsubparagraph{#1}\mbox{}}
\fi
% definitions for citeproc citations
\NewDocumentCommand\citeproctext{}{}
\NewDocumentCommand\citeproc{mm}{%
  \begingroup\def\citeproctext{#2}\cite{#1}\endgroup}
\makeatletter
 % allow citations to break across lines
 \let\@cite@ofmt\@firstofone
 % avoid brackets around text for \cite:
 \def\@biblabel#1{}
 \def\@cite#1#2{{#1\if@tempswa , #2\fi}}
\makeatother
\newlength{\cslhangindent}
\setlength{\cslhangindent}{1.5em}
\newlength{\csllabelwidth}
\setlength{\csllabelwidth}{3em}
\newenvironment{CSLReferences}[2] % #1 hanging-indent, #2 entry-spacing
 {\begin{list}{}{%
  \setlength{\itemindent}{0pt}
  \setlength{\leftmargin}{0pt}
  \setlength{\parsep}{0pt}
  % turn on hanging indent if param 1 is 1
  \ifodd #1
   \setlength{\leftmargin}{\cslhangindent}
   \setlength{\itemindent}{-1\cslhangindent}
  \fi
  % set entry spacing
  \setlength{\itemsep}{#2\baselineskip}}}
 {\end{list}}
\usepackage{calc}
\newcommand{\CSLBlock}[1]{\hfill\break\parbox[t]{\linewidth}{\strut\ignorespaces#1\strut}}
\newcommand{\CSLLeftMargin}[1]{\parbox[t]{\csllabelwidth}{\strut#1\strut}}
\newcommand{\CSLRightInline}[1]{\parbox[t]{\linewidth - \csllabelwidth}{\strut#1\strut}}
\newcommand{\CSLIndent}[1]{\hspace{\cslhangindent}#1}
\ifLuaTeX
\usepackage[bidi=basic]{babel}
\else
\usepackage[bidi=default]{babel}
\fi
\babelprovide[main,import]{english}
% get rid of language-specific shorthands (see #6817):
\let\LanguageShortHands\languageshorthands
\def\languageshorthands#1{}
% Manuscript styling
\usepackage{upgreek}
\captionsetup{font=singlespacing,justification=justified}

% Table formatting
\usepackage{longtable}
\usepackage{lscape}
% \usepackage[counterclockwise]{rotating}   % Landscape page setup for large tables
\usepackage{multirow}		% Table styling
\usepackage{tabularx}		% Control Column width
\usepackage[flushleft]{threeparttable}	% Allows for three part tables with a specified notes section
\usepackage{threeparttablex}            % Lets threeparttable work with longtable

% Create new environments so endfloat can handle them
% \newenvironment{ltable}
%   {\begin{landscape}\centering\begin{threeparttable}}
%   {\end{threeparttable}\end{landscape}}
\newenvironment{lltable}{\begin{landscape}\centering\begin{ThreePartTable}}{\end{ThreePartTable}\end{landscape}}

% Enables adjusting longtable caption width to table width
% Solution found at http://golatex.de/longtable-mit-caption-so-breit-wie-die-tabelle-t15767.html
\makeatletter
\newcommand\LastLTentrywidth{1em}
\newlength\longtablewidth
\setlength{\longtablewidth}{1in}
\newcommand{\getlongtablewidth}{\begingroup \ifcsname LT@\roman{LT@tables}\endcsname \global\longtablewidth=0pt \renewcommand{\LT@entry}[2]{\global\advance\longtablewidth by ##2\relax\gdef\LastLTentrywidth{##2}}\@nameuse{LT@\roman{LT@tables}} \fi \endgroup}

% \setlength{\parindent}{0.5in}
% \setlength{\parskip}{0pt plus 0pt minus 0pt}

% Overwrite redefinition of paragraph and subparagraph by the default LaTeX template
% See https://github.com/crsh/papaja/issues/292
\makeatletter
\renewcommand{\paragraph}{\@startsection{paragraph}{4}{\parindent}%
  {0\baselineskip \@plus 0.2ex \@minus 0.2ex}%
  {-1em}%
  {\normalfont\normalsize\bfseries\itshape\typesectitle}}

\renewcommand{\subparagraph}[1]{\@startsection{subparagraph}{5}{1em}%
  {0\baselineskip \@plus 0.2ex \@minus 0.2ex}%
  {-\z@\relax}%
  {\normalfont\normalsize\itshape\hspace{\parindent}{#1}\textit{\addperi}}{\relax}}
\makeatother

\makeatletter
\usepackage{etoolbox}
\patchcmd{\maketitle}
  {\section{\normalfont\normalsize\abstractname}}
  {\section*{\normalfont\normalsize\abstractname}}
  {}{\typeout{Failed to patch abstract.}}
\patchcmd{\maketitle}
  {\section{\protect\normalfont{\@title}}}
  {\section*{\protect\normalfont{\@title}}}
  {}{\typeout{Failed to patch title.}}
\makeatother

\usepackage{xpatch}
\makeatletter
\xapptocmd\appendix
  {\xapptocmd\section
    {\addcontentsline{toc}{section}{\appendixname\ifoneappendix\else~\theappendix\fi\\: #1}}
    {}{\InnerPatchFailed}%
  }
{}{\PatchFailed}
\keywords{meta-analysis, standard-operating-procedures, meta-science\newline\indent Word count: 2119}
\DeclareDelayedFloatFlavor{ThreePartTable}{table}
\DeclareDelayedFloatFlavor{lltable}{table}
\DeclareDelayedFloatFlavor*{longtable}{table}
\makeatletter
\renewcommand{\efloat@iwrite}[1]{\immediate\expandafter\protected@write\csname efloat@post#1\endcsname{}}
\makeatother
\usepackage{lineno}

\linenumbers
\usepackage{csquotes}
\ifLuaTeX
  \usepackage{selnolig}  % disable illegal ligatures
\fi
\usepackage{bookmark}
\IfFileExists{xurl.sty}{\usepackage{xurl}}{} % add URL line breaks if available
\urlstyle{same}
\hypersetup{
  pdftitle={Standard Operating Procedures for Meta-Analysis in the Paluck Lab},
  pdfauthor={Seth Green1, Elizabeth Levy Paluck1, \& Roni Porat2},
  pdflang={en-EN},
  pdfkeywords={meta-analysis, standard-operating-procedures, meta-science},
  hidelinks,
  pdfcreator={LaTeX via pandoc}}

\title{Standard Operating Procedures for Meta-Analysis in the Paluck Lab}
\author{Seth Green\textsuperscript{1}, Elizabeth Levy Paluck\textsuperscript{1}, \& Roni Porat\textsuperscript{2}}
\date{}


\shorttitle{Paluck\_meta\_SOP}

\authornote{

This work was supported by {[}whoever supports it{]}

The authors made the following contributions. Seth Green: Conceptualization, Writing - Original Draft Preparation, Writing - Review \& Editing; Elizabeth Levy Paluck: Writing - Review \& Editing, Supervision; Roni Porat: Writing - Review \& Editing, Supervision.

Correspondence concerning this article should be addressed to Seth Green, Kahneman-Treisman Center, Princeton University. E-mail: \href{mailto:sag2212@columbia.edu}{\nolinkurl{sag2212@columbia.edu}}

}

\affiliation{\vspace{0.5cm}\textsuperscript{1} Princeton University\\\textsuperscript{2} Hebrew University, Jerusalem}

\abstract{%
This paper describes the motivations and procedures for meta-analyses by Betsy Levy Paluck and co-authors, with reference and examples drawn from four such papers (Paluck, Green, and Green 2019, Paluck et al.~2021, and Gantman et al.~2024). We first describe our conceptual aims when writing meta-analytic papers, which are to provide an `on-ramp' to interested readers, to evaluate a literature's strengths and weaknesses, and to illuminate empirical gaps, which we intend as signposts for future researchers. Second, we describe the typical `flow' of the papers as a whole. Third, we describe our meta-analytic procedures, detailing our default reporting structure and analytic choices and engaging with recent critiques from Slough and Tyson (2022) and Simonsohn, Simmons, and Nelson (2022). Fourth, we introduce a collection of functions we use to implement our meta-analytic choices, available as an R package. Fifth, we conclude with some open questions for meta-analysts.
}



\begin{document}
\maketitle

\subsection{Introduction: Synthesizing meta-analysis and meta-science}\label{introduction-synthesizing-meta-analysis-and-meta-science}

What purposes do meta-analyses serve above and beyond those met by systematic reviews? Judging by recent papers in social psychology (\textbf{powerpose?}), the answer is to bolster and validate existing findings through aggregation. By pooling findings from many studies, researchers furnish meta-analytic estimates that are typically heartening to a literature's boosters by dint of their statistical precision and suggestion of broad applicability across settings. Such papers often have a valedictory quality to them: they provide leading practitioners a chance to survey the body of work they've helped inspire, highlight the strength of the evidence for their subfield's central tenets, and suggest future research directions that assume those tenets are no longer in need of further testing.

Meanwhile, a parallel movement in psychology has begun to implement the findings and suggestions of the open science (\textbf{nosek2012?}) and meta-science (whatever cite) into both study designs (\textbf{whatever?}) and literature reviews (\textbf{whatever?}). Some of those reviews provide quantitative assessments of meta-scientific quantitites, for instance, what percentage of a subset of papers are computationally reproducible (\textbf{objels2020?}) or consistent with open data standards (\textbf{whatever?}). In some sense, these estimates are meta-analytic, but they aren't meta-analyses in the conventional sense of pooling existing findings into average effects (\textbf{cochrane?}). In tone, these papers are often much more critical than meta-analyses are, and they are often written by outsiders or newcomers to an academic field rather than leading experts. Thus, though they share part of their name, the meta-analytic and meta-scientific tradition generally bring very different starting assumptions, use different tools, and are fundamentally aimed at different audiences:

At the Paluck lab, we seek to synthesize these two approaches by writing meta-analyses that are fundamentally meta-scientific. Over the course of four systematic reviews, three with meta-analytic components (\textbf{gantman2024?}), we've developed and honed somewhat idiosyncratic beliefs about where and when a meta-analysis is useful, how to structure our meta-analytic arguments, and default specifications. This paper articulates those beliefs, and aims to illuminate how meta-analysis might become a standard part of the meta-scientist's toolkit.

In short, we believe that meta-analyses are useful for evaluating the strengths and weaknesses of large, heterogeneous literatures, especailly those that bring mutliple theoretical perspectives to bear. They can also be useful for providing focused, critical re-examination of a literature's main findings as a contrast to previous credulous reviews. While some critics have argued that core assumptions about homogeneity across experimental contrasts and outcomes are necessary for the standard meta-analytic model to hold (\textbf{slough2022?}), we think that meta-analyses can still be useful when these conditions aren't met; in such cases, meta-analytic estimates should be treated as \emph{descriptive} rather than \emph{causal} inferences. Moreover, we see value in contrasting all-inclusive estimates, which are fundamentally descriptive, to more rigorous, focused estimates, which come about by imposing additional structure on which studies and outcomes are pooled and which are fundamentally causal. The contrasts in magnitude and precision between these two approaches are often startling and illuminating.

\subsubsection{Meta-analysis at 50: from social to medical science and back again}\label{meta-analysis-at-50-from-social-to-medical-science-and-back-again}

\begin{itemize}
\item
  Since the term was first coined in 1976 (Glass, 1976), meta-analysis has come to be understood as a crucial tool for synthesizing evidence in the medical and natural sciences (citation)
\item
  core idea, as a 1990 NYT article put put it, is to allow researchers ``to draw more reliable inferences or new conclusions from the combined results than from smaller studies that may be inconclusive individually'' (Altman, 1990)
\item
  labeled as controversial in that article, but the examples are all from medicine. then \href{https://www.nytimes.com/2012/10/16/science/stanford-organic-food-study-and-vagaries-of-meta-analyses.html}{debated vigorously in 2012}
\item
  take an excellent conventional meta-analysis, e.g. Meremikwu, Donegan, Sinclair, Esu, and Oringanje (2012). This study looks at Intermittent Preventive Treatment for malaria in seven trials, with 12,589 subjects,all of which ``were conducted in West Africa, and six of seven trials were restricted to children aged less than 5 years.'' The study finds that IPT leads to significant reductions in both malaria cases and severe malaria cases, ``a small reduction in all‐cause mortality,'' and that ``drug‐related adverse events, if they occur, are probably rare.''
\item
  Note four qualities of this meta-analysis:

  \begin{enumerate}
  \def\labelenumi{\arabic{enumi}.}
  \tightlist
  \item
    All RCTs
  \item
    Singular,cohesive treatment
  \item
    Seven different draws from a common population
  \item
    Coherent, commonly measured outcomes that clearly represent key quantities of interest
    In the past 20 years, meta-analysis has become much more common in the social sciences and in psychological science in particular. But if we look at some prominent ones, we might say that these conditions of the ``ideal'' meta-analysis don't really hold.
  \item
    Methodological diversity (Pettigrew and Tropp 2006 integrate studies of widely different designs)
  \item
    Very diverse treatments (\url{https://psycnet.apa.org/record/2016-43598-001})
  \item
    Diverse settings (cite example)
  \item
    Broad array of outcome measures (Paluck et al.~2020)
  \end{enumerate}
\end{itemize}

\subsubsection{Contemporary criticisms of meta-analysis from meta-scientists}\label{contemporary-criticisms-of-meta-analysis-from-meta-scientists}

Recent papers have raised the possibility that the core assumptions of meta-analysis are often not met in contemporary work.

Slough and Tyson (2022) argue that two core conditions are necessary for meta-analysis to make sense: contrast and target harmonization\ldots{}

Meanwhile, Simonsohn et al (SSN). argue\ldots{}

These critiques register deeply with us because of our three recent meta-analyses, two strive to be comprehensive and therefore arguably do not meet the assumptions laid out by Slough and Tyson and are guilty of the sins identified by SSN (though SSN also identify one of papers as exemplary)

So does meta-analysis still \emph{work} under these conditions? Is there something cogent and meaningful here?

\subsubsection{All models are wrong, but this one is still useful}\label{all-models-are-wrong-but-this-one-is-still-useful}

We say yes, we still think meta-analysis can be useful even if the underlying literatures do not meet the ideal conditions of a medical literature, with the qualification that meta-analytic estimates need to be presented and contextualized appropriately. We urge researchers not to throw the baby out with the bathwater -- as always, all models are wrong, but we think this one is useful.

This article outlines how we make our meta-analyses useful, and to whom. We start by identifying what makes a literature a good candidate for meta-analysis, our overarching purposes when we write one, and who we write such pieces for. We then describe our typical paper structure for a meta-analysis; our default analytic choices; and an R package, \texttt{PaluckMetaSOP}, that implements our defaults and helps aids our writing process. We conclude with some open questions and hard cases

{[}where does this go{]} this piece also contributes to two burgeoning literatures:
SOPs (\url{https://alexandercoppock.com/Green-Lab-SOP/Green_Lab_SOP.pdf}, \url{https://www.stat.berkeley.edu/~winston/sop-safety-net.pdf}) + ones Don cited
establishing theoretical underpinnings of meta-analysis (Slough and Tyson, SSN, Gechter and Meager)

\paragraph{The remainder of this paper, and what this paper is not}\label{the-remainder-of-this-paper-and-what-this-paper-is-not}

\begin{itemize}
\item
  Paper
\item
  What this paper is not: the Cochrane review
  4. The Cochrane review is still the seminal text, and this piece exists in dialogue with it.
\end{itemize}

\paragraph{Notes}\label{notes}

\begin{enumerate}
\def\labelenumi{\arabic{enumi}.}
\tightlist
\item
  Meta-analyses are common in the social sciences -- cite some big ones.
\item
  And yet, meta-analyses are surprisingly undertheorized.

  \begin{enumerate}
  \def\labelenumii{\arabic{enumii}.}
  \tightlist
  \item
    Most of the guides for meta-analysis are aimed at medical doctors and researchers and presume that you contrast and target harmonization (Slough and Tyson 2022) -- in other words, that the ``X'' and the ``Y'' in ``does X influence Y?'' are basically the same across studies, and the goal is just to average their results.

    \begin{enumerate}
    \def\labelenumiii{\arabic{enumiii}.}
    \tightlist
    \item
      Sometimes that holds, e.g.~\href{https://scholar.harvard.edu/files/kremer/files/meta-analysis_deworming_world_bank_working_paper_dec_2016.pdf}{deworming} or \href{https://www.mmv.org/sites/default/files/uploads/docs/access/SMC_Tool_Kit/publications/Meremikww-ipt-review.pdf}{insecticide-treated bednets}
    \item
      But in the behavioral sciences, the inputs are probably a lot more heterogeneous, e.g.~\href{https://psycnet.apa.org/record/2016-43598-001}{diversity} \href{https://compass.onlinelibrary.wiley.com/doi/10.1111/spc3.12741?af=R}{training} or \href{https://pubmed.ncbi.nlm.nih.gov/16737372/}{the} \href{https://psycnet.apa.org/record/2015-07056-001}{contact} \href{https://journals.sagepub.com/doi/abs/10.1177/1088868318762647}{hypothesis}, as are the outputs, as are the outputs, e.g.~a mix of attitudinal and behavioral outcomes and sometimes totally bespoke instruments.
    \item
      Looking at \href{https://training.cochrane.org/handbook/current/chapter-10}{the Cochrane Review}, we see guidelines that are thorough, but generally propose statistical solutions to what are fundamentally non-statistical sources of uncertainty. By contrast, \href{https://www.nature.com/articles/s44159-022-00101-8}{Simmonsohn et al.~(2022)} argue for \emph{design}-based solutions.
    \end{enumerate}
  \item
    Over the course of 8 years and three meta-analyses, we've developed a distinct perspective about when to do meta-analysis, what it's useful for, and how to do it. This article articulates that perspective.
  \item
    SOPs (\url{https://alexandercoppock.com/Green-Lab-SOP/Green_Lab_SOP.pdf}, \url{https://www.stat.berkeley.edu/~winston/sop-safety-net.pdf})
  \end{enumerate}
\item
  So when are meta-analyses useful?
\end{enumerate}

\subsection{When are meta-analyses useful and value do they bring?}\label{when-are-meta-analyses-useful-and-value-do-they-bring}

\subsubsection{When do we do meta-analyses?}\label{when-do-we-do-meta-analyses}

\begin{enumerate}
\def\labelenumi{\arabic{enumi}.}
\tightlist
\item
  Our lab embarks on meta-analyses when

  \begin{enumerate}
  \def\labelenumii{\arabic{enumii}.}
  \tightlist
  \item
    Big literature with heterogeneous research quality
  \item
    Multiple theoretical approaches whose comparative efficacy is unknown
  \item
    Intuition that a comprehensive read through and tests for rigor will reveal collective gaps in understanding
  \end{enumerate}
\item
  Alternatively, if someone else has already written a comprehensive review, there might be room for a focused rejoinder, i.e.~Pettigrew and Tropp (2006) amalgamate everything and find large effects, whereas Paluck Green and Green (2019) look at just the best studies and find much more mixed results.
\end{enumerate}

\subsubsection{What purposes do they serve?}\label{what-purposes-do-they-serve}

\begin{enumerate}
\def\labelenumi{\arabic{enumi}.}
\tightlist
\item
  On-ramp to unfamiliar literature (alternate metaphor: ``a lobby,'' from Toni Morrison's introduction to Beloved), exemplified by

  \begin{enumerate}
  \def\labelenumii{\arabic{enumii}.}
  \tightlist
  \item
    Contact hypothesis: what, where, why
  \item
    \begin{enumerate}
    \def\labelenumiii{\roman{enumiii}.}
    \setcounter{enumiii}{2}
    \tightlist
    \item
      Prejudice reduction: theoretical perspectives, landmark studies
    \end{enumerate}
  \item
    \begin{enumerate}
    \def\labelenumiii{\roman{enumiii}.}
    \setcounter{enumiii}{3}
    \tightlist
    \item
      Sexual violence paper: history of zeitgeist perspectives
    \end{enumerate}
  \end{enumerate}
\item
  Evaluate rather than transcribe

  \begin{enumerate}
  \def\labelenumii{\arabic{enumii}.}
  \setcounter{enumii}{3}
  \tightlist
  \item
    Describe critique of transcriptive, non-evaluative work in Simonsohn et al.~(2022)
  \item
    We identify high points (excellent studies) and low points (widespread methodological deficiencies)
  \item
    place effect sizes in context of real-world impact whenever possible, e.g.~the prejudice paper argues that \(\Delta\) of 0.27 corresponds to {[}X{]} change on ANES survey
  \end{enumerate}
\item
  Illuminate empirical gaps

  \begin{enumerate}
  \def\labelenumii{\arabic{enumii}.}
  \setcounter{enumii}{6}
  \tightlist
  \item
    As our reviews show, applying some fairly minimal quality standards quickly winnows hundreds of studies down to just a few
  \item
    These studies are likely to be the most policy-relevant, and oftentimes the `ideal' study simply hasn't been conducted at all yet
  \item
    Examples:

    \begin{enumerate}
    \def\labelenumiii{\arabic{enumiii}.}
    \tightlist
    \item
      Paluck, Green and Green (2019) highlight lack of studies testing interracial contact among adults -\textgreater{} provide intellectual backdrop for Scacco and Warren (2018), Mousa (2020), and Lowe (2019) -- as well as providing theoretical underpinning for those studies' comparatively underwhelming findings
      2 similar examples from Paluck et al.~(2021)?
    \item
      (arguably a systematic review suffices for this)
    \end{enumerate}
  \end{enumerate}
\end{enumerate}

\subsubsection{Who are they for?}\label{who-are-they-for}

\begin{enumerate}
\def\labelenumi{\arabic{enumi}.}
\item
  researchers, especially folks trying to figure out where they can have the most impact
\item
  policymakers and funders who want a critical review of a literature
\item
  any interested reader. Our metas tend to be about big, important subjects that people are thinking about.
  \#\# Typical structure of Paluck lab meta-analytic papers
\item
  Describe context and stakes of the paper (i.e.~why it's important to see if these interventions `work', or why there's reason to doubt an established consensus)
\item
  Intellectual overview of major ideas in the literature
\item
  Meta-analytic search methods
\item
  Meta-analytic approach
\item
  Quantitative Results
\item
  Discussion, highlights and gaps
\item
  Conclusion
\end{enumerate}

\subsection{Paluck lab meta-analytic procedures}\label{paluck-lab-meta-analytic-procedures}

\begin{enumerate}
\def\labelenumi{\arabic{enumi}.}
\tightlist
\item
  Code more than you think you're going to need

  \begin{enumerate}
  \def\labelenumii{\arabic{enumii}.}
  \tightlist
  \item
    You might be interested in the lasting effects, which means recording the latest possible effect sizes, but if you record the \emph{earliest} possible effects, you can detect within-study decay
  \end{enumerate}
\item
  Start big and then go small, like a funnel; i.e.~meta-analyze \emph{everything} and then zoom in on different subsets of the literature

  \begin{enumerate}
  \def\labelenumii{\arabic{enumii}.}
  \setcounter{enumii}{1}
  \tightlist
  \item
    Why meta-analyze everything?

    \begin{enumerate}
    \def\labelenumiii{\arabic{enumiii}.}
    \tightlist
    \item
      Slough and Tyson (2023) argue that meta-analytic estimates only make sense if there's contrast harmony, i.e.~if the many studies are clearly testing some common underlying theoretical concept. In some literatures we've looked at, this pretty clearly does not hold, e.g.~the average effect of a contact hypothesis field experiment for Muslims and Christians and Nigeria and a diversity training in a U.S. corporate workplace tells you the combined effect of \emph{what} exactly?
    \item
      Ditto with ``target harmony'', which is whether interventions are aimed \emph{at} changing the same thing. For instance, does it make sense to average the effect of some intervention on self-reported attitudes and behavioral outcomes? As Gantman et al.~(2023) show, these outcomes are not always well correlated, which is prima facie evidence that we don't have target harmony.
    \item
      Last, integrating observational and randomized studies is often justified as supplementing high studies with high internal validity with studies with high \emph{external} validity, but it's probably more accurate to say that you take some studies that provide unbiased estimates and then, in exchange for greater statistical precision, induce bias arising from non-statistical sources of uncertainty (\href{http://www.donaldgreen.com/wp-content/uploads/2015/09/Gerber-Green-Kaplan-IllusionofLearning.pdf}{Gerber, Green and Kaplan 2004}).
    \end{enumerate}
  \item
    multi-part answer.

    \begin{enumerate}
    \def\labelenumiii{\arabic{enumiii}.}
    \setcounter{enumiii}{3}
    \tightlist
    \item
      Overall meta-analytic estimate provides a within-paper size comparison; your overall effect size is X and your effect size for the very best studies is 1/3 X, that means something.
    \item
      Test for publication bias should probably look at absolutely everything
    \item
      Inter-paper comparisons; Paluck, Green and Green (2019) and Paluck et al.~(2021) provide estimates of about \(\Delta\) = 0.3, and then Green, Smith and Mathur (forthcoming) find \(\Delta\) = 0.138, which also means something.
    \end{enumerate}
  \end{enumerate}
\item
  Do some serious tests for publication bias: both conventional tests (egger's test, funnel plot, \(\Delta\) \textasciitilde{} SE), but also think through where else it might emerge, e.g.~compare effect sizes in studies with and without DOIs
\item
  separate the dataset into different chunks, e.g.~by theory, study design, or measurement strategy, and present these meta-analytic estimates side by side

  \begin{enumerate}
  \def\labelenumii{\arabic{enumii}.}
  \setcounter{enumii}{3}
  \tightlist
  \item
    Subset enough and eventually you get to reasonable claim for contrast and mechanism harmony (Slough and Tyson 2022)
  \end{enumerate}
\item
  Zoom in on and carefully and analyze best studies, both in aggregate and individually

  \begin{enumerate}
  \def\labelenumii{\arabic{enumii}.}
  \setcounter{enumii}{4}
  \tightlist
  \item
    These results might be surprising, e.g.~when looking solely at RCTs on perpetration outcomes (Gantman et al.~2024), slightly fewer than half are self-reported nulls.
  \end{enumerate}
\item
  Meta-analytic defaults

  \begin{enumerate}
  \def\labelenumii{\arabic{enumii}.}
  \setcounter{enumii}{5}
  \tightlist
  \item
    Random effects: any literature we want to look at is going to have heterogeneous inputs.
  \item
    Cluster at level of study
  \item
    Glass's \(\Delta\) rather than Cohen's d
  \item
    Difference in proportion rather than odds ratio

    \begin{enumerate}
    \def\labelenumiii{\arabic{enumiii}.}
    \setcounter{enumiii}{6}
    \tightlist
    \item
      resurface text from appendix to Gantman et al.~(2024)
    \end{enumerate}
  \end{enumerate}
\end{enumerate}

\subsection{R package: blp\_meta\_functions}\label{r-package-blp_meta_functions}

\begin{enumerate}
\def\labelenumi{\arabic{enumi}.}
\tightlist
\item
  Functions fall into four categories

  \begin{enumerate}
  \def\labelenumii{\arabic{enumii}.}
  \tightlist
  \item
    Converting studies to singular estimates of effect size, variance and standard error
  \item
    Wrapper functions that distill aggregate results into the core findings and make them table-ready
  \item
    plotting functions
  \item
    Miscellany: reproducibility, helper functions
  \end{enumerate}
\end{enumerate}

\subsection{Conclusions: hard cases}\label{conclusions-hard-cases}

\begin{enumerate}
\def\labelenumi{\arabic{enumi}.}
\tightlist
\item
  What do we do when a paper's results are obviously not credible?

  \begin{enumerate}
  \def\labelenumii{\arabic{enumii}.}
  \tightlist
  \item
    e.g.~if they present a result that's well past possible, e.g.~t-test value of 36 (one of the papers in the prejudice literature had this, and also the cost-benefit analyses in the SOSA! Intervention in the primary prevention literature)
  \end{enumerate}
\item
  What's the right outcome to code? Often very unclear what's prime or most representative of underlying construct
\item
  Others\ldots{}
\item
  Meta-analyses are still very useful, we think, for readers (both expert and non-expert) and future researchers.
\end{enumerate}

\newpage

\phantomsection\label{refs}
\begin{CSLReferences}{1}{0}
\bibitem[\citeproctext]{ref-altman1990}
Altman, L. K. (1990). New method of analyzing health data stirs debate. \emph{New York Times}, 31.

\bibitem[\citeproctext]{ref-glass1976}
Glass, G. V. (1976). Primary, secondary, and meta-analysis of research. \emph{Educational Researcher}, \emph{5}(10), 3--8.

\bibitem[\citeproctext]{ref-meremikwu2012}
Meremikwu, M. M., Donegan, S., Sinclair, D., Esu, E., \& Oringanje, C. (2012). Intermittent preventive treatment for malaria in children living in areas with seasonal transmission. \emph{Cochrane Database of Systematic Reviews}, (2).

\end{CSLReferences}


\end{document}
